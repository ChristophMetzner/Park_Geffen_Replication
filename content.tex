\subsection{Introduction}


Transcranial electrical and magnetic stimulation  (TES and TMS, respectively) have been increasingly used in studies over the last decades and have been found to
alter and enhance cognitive processes
\supercite{Walsh2000,Lage2016,Paulus2016}. Furthermore, deep brain stimulation (DBS) has shown remarkable results in the treatment of tremor symptoms in 
Parkinsons disease \supercite{Collomb-Clerc2015} and also great potential 
in the treatment of psychiatric disorder such as obsessive-compulsive disorder \supercite{Alonso2015}.
However, despite this growing success in clinical settings, a principled understanding of the effects of stimulation on the dynamical processes in the brain 
is still lacking and, hence, stimulation parameters and target areas are currently not being optimized in a systematic fashion.    

Therefore, Muldoon et al. \supercite{Muldoon2016} develop a framework to explore the effects of targeted transcranial or deep brain stimulation on overall 
brain dynamics. In their framework they use data-driven computational model based on subject-specific structural connectivity and a nonlinear model of regional
brain activity (the so-called Wilson-Cowan model \supercite{Wilson1972}). Furthermore, they demonstrate that structure-based measures from linear network control theory 
can predict the functional effect of targeted stimulation.

In this work, we present an implementation of the modelling framework from Muldoon et al. \supercite{Muldoon2016} writtem im pure Python, where we exchanged
the model of regional brain activity to a simpler, phenomenological model, the FitzHugh Nagumo model \supercite{FitzHugh1961}. We report a \textit{partial/full(?)} 
replication of their results.
\subsection{Methods}

\paragraph{Overview}
The original model is, to the best of our knowledge, not publicly available. Therefore, this reimplementation is purely based on the description in the original
article and its supplementary material. The reimplementation is written in Python (2.7/3.6? tested with the other?). 

We have made two major changes compared to the original article: First, because we did not have access to the DTI data set used in the article, we used publicly
available DTI data form the Human Connectome Project \supercite{}. Second, we replaced the Wilson-Cowan model \supercite{Wilson1972} by the simpler and 
more efficient FithHugh Nagumo model \supercite{FitzHugh1961}. Both changes will be explained in more detail in the following sections and commented on in the 
discussion.
\paragraph{Structural connectivity}
As explained above, we did not have access to the diffusion tensor imaging (DTI) data form the original article. We thus used DTI data from 10 healthy
subjects (mean+std age and gender distribution; possibly even more demographic info \textbf{\textit{Nico!}}) from the publicly available HCP1200 data set of the
Human Connectome Project (\textit{link}).

\begin{itemize}
 \item 
  Acquisition information from HCP
 \item
  Preprocessing: software tool, steps and parameters  \textbf{\textit{Nico! Maybe Cristiana?}}
 \item
  Structural connectivity matrices: Parcellation scheme, thresholding, normalization \textbf{\textit{Nico!}}
\end{itemize}

\textit{Do we have repeated scans of the same individuals? If not, we cannot look at Reproducibility and the within subject
variability (see Figure 2 d). Not a big problem, but should then be mentioned here!}

\paragraph{Model of regional brain dynamics}
The second part that is necessary to model the dynamic evolution of brain activity across the entire brain, is a computational model of the
regional brain dynamics. In the original article, Muldoon et al. \supercite{Muldoon2016} use the Wilson-Cowan model \supercite{Wilson1972}. The Wilson-Cowan model
describes the interaction between a local excitatory and a local inhibitory neural population by two coupled differential equations. However, due to the different
parameters describing the coupling of the populations and the synaptic interactions, the Wilson-Cowan model has more then ten free parameters and, thus,
a high degree of unconstrained variability. Therefore, we opted to exchange the Wilson-Cowan model by the FitzHugh Nagumo (FHN) model \supercite{FitzHugh1961}, which
only has six (\textit{correct?}\textbf{\textit{Nico!}}) free parameters and, additionally, is computationally slightly more efficient than the Wilson-Cowan model.

The FHN is given by...\textit{  FHN equations and parameters (plus explanation)} \textbf{\textit{Nico!}}


Importantly, the FHN model displays the same three dynamical states depending on the strength of its input than the Wilson-Cowan model (see Figure \ref{fig:FHN}): 
First, for weak input the FHN resides in a stable fixed point with low activity. Second, for stronger inputs the FHN then transitions into an oscillatory
regime and, last, for even stronger inputs it then settles into a stable fixed point again, however, this time with a very high activity.

\begin{figure}
\includegraphics[width=\textwidth]{Figures/FHN_dynamics}
 \caption{ 3 states of the FitzHugh Nagumo model. Analguous to the Wilson-Cowan model, depending on the strength of the external input, the FitzHugh Nagumo
 model displays either a low fixed point state (left panel), a sustained oscillation (middle panel), or a high activity fixed point (right panel).}
 \label{fig:FHN}
\end{figure}


\paragraph{Individual oscillatory transition paramters}
\textbf{\textit{Nico!}}



\paragraph{Linear network control theory}
Muldoon et al. \supercite{Muldoon2016} utilize the control theoretic notion of controllability to get theoretical insight inot the effect stimulation might have.
They choose two different measures based on a linear control setting: \textit{average controllability} and \textit{modal controllability}. Average controllability 
is given by the average input energy to the control nodes over all possible target states. Typically, nodes with high average controllability can control networks 
dynamics over nearby target states in an energy efficient way. On the other modal controllability decribes how well a node can control all network modes. High modal 
controllability means that such a node is able to reach all modes of a network and, thus, can force the dynamics into hard-to-reach target states.

\paragraph{Functional and structural effect, and fractional activation}
We calculate functional connectivity matrices as pairwise cross-correlation between the $x_1$ components of the FHN model of each node. Throughout, we 
use a 1s time window. As in the original article, the stimulation trials consist of three phases each comprising 1s: the first phase, without any external input, 
allows the model to stabilise its activity, the second phase, again without external input, is used as a baseline and, lastly, in the third phase the stimulation
is applied, i.e. the external input parameter of a single node $i$ is set to $I_i(t)=0.52?$ \textbf{\textit{Nico}} while all others remain 0. 

\subsection{Reproduction of experiments}




\paragraph{Regional controllability}
Replicating the findings of Muldoon et al. \supercite{Muldoon2016}, we see that average and modal controllability strongly correlate with node degree for our
data set of structural connectivity matrices. While average 
controllability and degree are positively related(\ref{fig:controllability-measures} (a)), modal controllability typically decreases with increasing node degree 
(\ref{fig:controllability-measures} (b)).

\begin{figure}
\subfigure[]{
\includegraphics[width=\textwidth]{Figures/average_controllability}}
\subfigure[]{
\includegraphics[width=\textwidth]{Figures/modal_controllability}}
 \caption{\textbf{Regional controllability.} (a) Relationship between node degree  and (a) average controllability and (b) modal controllability, respectively, 
 for all regions and subjects.\textit{replicates Figure 3}} 
 \label{fig:controllability-measures}
\end{figure}

Using these structural connectivity matrices, we then simulated the effect of electrical stimulation on changes in the global network dynamics. We followed the 
simulation protocol from the original article \supercite{Muldoon2016}: We first chose the global coupling strength such that all regional nodes are just below the 
transition from the low fixed point to the oscillatory regime. Then, a single region is chosen and an external current is applied to this region such that this
region transitions to the oscillatory regime. By calculating the functional connectivity matrix (the matrix containing the pairwise correlations between regions) 
for the model before and after stimulation and substracting the former from the latter, we can measure the functional effect of this particular stimulation. Again, 
replicating the original findings, we find that regions with low average controllability led to small changes in functional connectivity (an example can be seen
in \ref{fig:functional-effect-examples} (a)) while regions with high average controllability led to larger changes in functional connectivity 
\ref{fig:functional-effect-examples} (b)).

\begin{figure}
\subfigure[]{
\includegraphics[width=0.8\textwidth]{Figures/DeltaFC25}}
\subfigure[]{
\includegraphics[width=0.8\textwidth]{Figures/DeltaFC70}}
 \caption{\textbf{Effect of regional stimulation.} Effect of stimulation of two example regions: (a) Region 25 (name?) with a low average controllability (value?),
 (b) Region 70 (name?) with a high average controllability (value?). \textit{replicates Figure 4c}} 
 \label{fig:functional-effect-examples}
\end{figure}



\begin{figure}
\subfigure[]{
\includegraphics[width=0.4\textwidth]{Figures/functional_effect_modal}}
\subfigure[]{
\includegraphics[width=0.4\textwidth]{Figures/functional_effect_modal}}\\
\subfigure[]{
\includegraphics[width=0.4\textwidth]{Figures/functional_effect_modal}}
\subfigure[]{
\includegraphics[width=0.4\textwidth]{Figures/functional_effect_modal}}
 \caption{\textbf{Functional effect and controllability.} (a-b) Functional effect of nodal stimulation plotted against average (a) and modal (b) controllability.
 \textit{replicates Figure 5; change b-d; redo all figures with individual subject resolution }\textbf{\textit{Questions: Why do we have modal controllability 
 ranging from 0.4 to 1 while Muldoon from 0.965 to 1; compare to Figure 2???? Why is our functional higher than theirs ([0.6-0.9] compared to [0-0.8]?}}} 
 \label{fig:controllability-functional-effect}
\end{figure}



\begin{itemize}
 \item 
  Replicate Figure 2 (c) and (d) 
  and for c) our box plots for all subjects, and d) bar plots with our data
  
 \item
  Replicate Figure 5 b)-d) (plot avg controllability vs. functional effect; calculate structural effect and plot vs avg/modal controllability)
  
 \item
  Replicate Figure 6 (?) (calculate fractional activation and plot against functional/structural effect; plot fraction activation FCs for node 25 and 70)
\end{itemize}






\subsection{Reimplementation}

\begin{itemize}
 \item 
  Details on the new implementation (packages/dependencies, other stuff?, maybe just a paragraph in the methods section) 
\end{itemize}

 




\subsection{Discussion}

\begin{itemize}
 \item 
  Main similarities and differences between our and original results. Replication: full, partial or not at all?
\end{itemize}
